\documentclass[a4paper,oneside,openany]{tufte-book}
\usepackage[utf8x]{inputenc}
\usepackage[german]{babel}
\usepackage{microtype} % Improves character and word spacing
\usepackage{booktabs} % Better horizontal rules in tables
\usepackage{ucs}
\usepackage{amsmath}
\usepackage{amsfonts}
\usepackage{amssymb}
\usepackage{graphicx}
\usepackage{color}
\usepackage{listings}
\usepackage{caption}

\makeatletter% since we're using commands with @ in their name

\let\origappendix\appendix% save a copy of the original meaning of \appendix
\renewcommand{\appendix}{%
  \origappendix% do all the original \appendix stuff
  \titlecontents{chapter}%
    [0em] % distance from left margin
    {\vspace{1.5\baselineskip}\begin{fullwidth}\LARGE\rmfamily\itshape} % above (global formatting of entry)
    {\hspace*{0em}\appendixname~\thecontentslabel: } % before w/label (label = ``2'')
    {\hspace*{0em}} % before w/o label
    {\rmfamily\upshape\qquad\thecontentspage} % filler + page (leaders and page num)
    [\end{fullwidth}] % after
  \titleformat{\chapter}%
    [display]% shape
    {\relax\ifthenelse{\NOT\boolean{@tufte@symmetric}}{\begin{fullwidth}}{}}% format applied to label+text
    {\itshape\huge Anhang~\thechapter}% label
    {0pt}% horizontal separation between label and title body
    {\huge\rmfamily\itshape}% before the title body
    [\ifthenelse{\NOT\boolean{@tufte@symmetric}}{\end{fullwidth}}{}]% after the title body
  \setcounter{secnumdepth}{0}% ``number'' the appendices
  \renewcommand{\thefigure}{\@arabic\c@figure}% define \thefigure to use only the figure number (1), not A.1
  \renewcommand{\thetable}{\@arabic\c@table}%
  %
  % Add any other special appendix-related code here.
  %
}
\makeatother% restore the special meaning of @


\author{Schett Matthias}
\title{SEN-\"{U}bung 10}

\begin{document}

\definecolor{gray}{rgb}{0.9,0.9,0.9} % background color for the listings

\lstset{language=[Visual]Basic, morekeywords={param, local}, breaklines=true, tabsize=4, showstringspaces=false,backgroundcolor=\color{gray}, numbers=left,
    basicstyle=\ttfamily} % standard listing settings

\frontmatter

\maketitle
\tableofcontents
\mainmatter

\chapter{Aufgabe 1}

\section{L\"{o}sungsidee}

Es werden drei Funktoren erstellt, CharacterComparator\sidenote{Vergleicht zwei einzelene Character auf gleicheit mittels toLower() um Klein/Großschreibung zu ignorieren}, LetterFrequency\sidenote{Zählt mittels count\_if und CharacterComparator die Anzahl an zu suchenden Charactern} und CompareByLetterFrequency\sidenote{Vergleicht und sortiert die Ergebnisse von LetterFrequency nach der Häufigkeit aufsteigend}.

Der Quellcode ist im Anhang unter \nameref{chap:Auf1} zu finden.

\section{Testf\"{a}lle}

\lstinputlisting[caption=Testtreiber]{LetterFrequency/OutputA1.txt}

\chapter{Aufgabe 2}

\section{L\"{o}sungsidee}

Der Funktor soll in Verbindung mit std::generate funktionieren, daher wird der Funktor ohne die Übergabe von Argumenten aufgerufen. Der Konstruktor bekommt, allerdings einen Startwert
und eine Schrittweite. Der Funktor gibt anschließend immer den Startwert += Schrittweite aus.

Der Quellcode ist im Anhang unter \nameref{chap:Auf2} zu finden.

\section{Testf\"{a}lle}

\lstinputlisting[caption=Testtreiber]{SequenceGenerator/OutputA2.txt}

\chapter{Aufgabe 3}

\section{L\"{o}sungsidee}

In einem Vector mit strings soll die maximale, mittlere und kürzeste String Länge mittels einen std::for\_each Aufrufes und eines Funktors realisiert werden.
Dazu wird der Funktor StringLength implementiert. Dieser hat folgende Member

\begin{itemize}
	\item maxLength
    \item minLength
    \item averageLenght
    \item numOfStrings
\end{itemize}

std::for\_each ruft den Funktor für jedes Element innerhalb der Iterator Range einzeln auf, daher wird nur ein einzelener String in den Funktor übergeben, dieser wird analysiert.
Abschließend liefert std::for\_each eine Kopie des Funktors zurürck, auf dieser Kopie wird dann die Print() Funktion aufgerufen, um die Statistik anzuzeigen.
Der Quellcode ist im Anhang unter \nameref{chap:Auf3} zu finden.

\section{Testf\"{a}lle}

\lstinputlisting[caption=Testausgabe]{StringLength/OutputA3.txt}

\backmatter

\appendix

\setboolean{@mainmatter}{true}
\chapter{Aufgabe 1}\label{chap:Auf1}

\begin{fullwidth}
\lstinputlisting[caption=Implementierung]{LetterFrequency/LetterFrequency.h}
\lstinputlisting[caption=Testtreiber]{LetterFrequency/Main.cpp}
\end{fullwidth}

\chapter{Aufgabe 2}\label{chap:Auf2}

\begin{fullwidth}
\lstinputlisting[caption=Testtreiber]{SequenceGenerator/Main.cpp}
\lstinputlisting[caption=Testtreiber]{SequenceGenerator/SequenceGenerator.h}
\end{fullwidth}

\chapter{Aufgabe 2}\label{chap:Auf2}

\begin{fullwidth}
\lstinputlisting[caption=Testtreiber]{StringLength/Main.cpp}
\lstinputlisting[caption=Testtreiber]{StringLength/StringLength.h}
\end{fullwidth}


\end{document}
